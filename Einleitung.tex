% Zusammenfassung

\pagenumbering{arabic}
\chapter{Einleitung}


Am modernen Finanzmarkt gewinnen Derivate immer mehr an Bedeutung und Handelsvolumen. Derivate sind Finanzinstrumente, deren Wert über den Wert und die Preisentwicklung eines anderen handelbaren Gutes bestimmt werden (lat. \textit{derivare} = \glqq ableiten\grqq). Gezielt gewählte Derivate bilden damit eine gute Möglichkeit, sich gegen Schocks oder ungünstige Kursentwicklungen des zugrundeliegenden Gutes abzusichern und gewinnen so ihre Beliebtheit.
Im Rahmen dieser Arbeit wollen wir uns Optionen anschauen. Optionen können auf verschiedene Güter (sogenannte Basiswerte) und zu vielen verschiedenen Konditionen geschrieben sein. Wir werden \glqq Europäische und Amerikanische Call- und Put-Optionen\grqq\,, deren Basiswert eine Aktie ist, anschauen und Modelle und Methoden herleiten, um den Preis -- den Betrag, der bei Vertragsschluss ausgetauscht wird -- dieser Derivate zu bestimmen. Natürlich bestimmt sich der tatsächliche Preis dieser Optionen nicht nach mathematischen Formeln und Gleichungen, sondern über Angebot und Nachfrage am Markt. Dennoch bilden die in dieser Arbeit vorgestellten Modelle (das Binomial- und das Black-Scholes-Modell) stets eine Orientierung für alle Markteilnehmer.

\vspace{0.8cm}
{\large\bf{Zusammenfassung}}

Zunächst wird im 2. Kapitel auf den wirtschaftlichen Hintergrund und die mathematischen Grundlagen eingegangen, welche für die Arbeit von zentraler Bedeutung sind. Anschließend werden wir in Kapitel 3 das zeitdiskrete Binomialmodell herleiten und auf dessen Konvergenz eingehen, um anschließend unsere Überlegungen teils auf das 4. Kapitel zu übertragen. In diesem werden wir auf das kontinuierliche Black-Scholes-Modell übergehen werden und dort zunächst, sofern möglich, einen analytischen Ansatz zur Optionsbewertung wählen. Da es allerdings für die Amerikanschen Optionen keinen analytischen Ansatz gibt, werden wir diese mithilfe numerischer Methoden (angelehnt an \cite{GuentherJuengel}) approximieren, wodurch wir auch Algorithmen für die Europäischen Optionen erhalten werden. Zuletzt werden wir uns in Kapitel 5 die hergeleiteten Algorithmen und deren Korrektheit an kurzen Beispielen anschauen, deren Implementierung im Anhang zu finden sind.